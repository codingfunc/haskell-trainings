%%%%%%%%%%%%%%%%%%%%%%%%%%%%%%%%%%%%%%%%%%%%%%%%%%%%%%%%%%%%%%%%%%%%%%%%%%%%
%% Header

\documentclass[17pt]{beamer}

\usepackage[english]{babel}
\usepackage{amssymb}
\usepackage{beamerthemesplit}
\usepackage{calc}
\usepackage{colortbl}
\usepackage{fdsymbol}
\usepackage{floatflt}
\usepackage[no-math]{fontspec}
\usepackage{graphicx}
\usepackage{listings}
\usepackage{soul}
\usepackage{tikz}
\usepackage{xunicode}

% When working on specific pages it is faster to generate just those pages.
% \usepackage[163-180]{pagesel}

\title{Haskell 102}
\author{pravnar@, javran@, mihaimaruseac@, ibobyr@, nicuveo@}
%\institute{Google}
\date{\small\today}



%%%%%%%%%%%%%%%%%%%%%%%%%%%%%%%%%%%%%%%%%%%%%%%%%%%%%%%%%%%%%%%%%%%%%%%%%%%%
%% Customization

%%
%% theme.tex
%% Made by nicuveo <antoine.jp.leblanc@gmail.com>
%%


%%%%%%%%%%%%%%%%%%%%%%%%%%%%%%%%%%%%%%%%%%%%%%%%%%%%%%%%%%%%%%%%%%%%%%%%%%%%
%% Base theme

\mode<presentation>{}
\usecolortheme{whale}
%\usepackage{fontspec}
\usepackage{etoolbox}
\setsansfont[Ligatures=TeX, % recommended
             UprightFont={*-Regular},
             ItalicFont={*-Regular},
             BoldFont={*-Bold},
             BoldItalicFont={*-Bold}]
             {Yanone Kaffeesatz}

%%%%%%%%%%%%%%%%%%%%%%%%%%%%%%%%%%%%%%%%%%%%%%%%%%%%%%%%%%%%%%%%%%%%%%%%%%%%
%% Colors

% define

\definecolor{google-r}{RGB}{234,  67,  53}
\definecolor{google-g}{RGB}{ 52, 168,  83}
\definecolor{google-b}{RGB}{ 66, 133, 244}
\definecolor{google-y}{RGB}{251, 188,   5}

\definecolor{my-color-1}{RGB}{ 66, 133, 244} % primary
\definecolor{my-color-2}{RGB}{234,  67,  53} % lighter
\definecolor{my-color-3}{RGB}{  0,   0,   0} % unknown
\definecolor{my-color-4}{RGB}{  0,   0,   0} % darker

\definecolor{tab-1}{rgb}{0.04,0.34,0.58}
\definecolor{tab-2}{rgb}{0.36,0.56,0.72}
\definecolor{tab-3}{rgb}{0.68,0.78,0.86}


% set

\setbeamercolor{structure}{fg=my-color-1}
\setbeamercolor{alerted text}{fg=my-color-2}

\setbeamercolor{palette primary}   {fg=white, bg=my-color-1}
\setbeamercolor{palette secondary} {fg=black, bg=my-color-2}
\setbeamercolor{palette tertiary}  {fg=white, bg=my-color-3}
\setbeamercolor{palette quaternary}{fg=white, bg=my-color-4}

\setbeamercolor{page number in head/foot} {fg=black, bg=white}
\setbeamercolor{icon in head/foot} {fg=black, bg=white}

\setbeamercolor*{separation line}{}
\setbeamercolor*{fine separation line}{}



%%%%%%%%%%%%%%%%%%%%%%%%%%%%%%%%%%%%%%%%%%%%%%%%%%%%%%%%%%%%%%%%%%%%%%%%%%%%
%% Templates

\newtoggle{showpagenumber}

\setbeamertemplate{headline}[default]
\defbeamertemplate{footline}{mine}[2]
{
  \leavevmode%
  \hbox
  {%
    \hskip .020\paperwidth

    \begin{beamercolorbox}[wd=.160\paperwidth,ht=3ex,dp=3ex,center]{icon in head/foot}
      \pgfimage[mask=#2,interpolate=true,height=14pt]{#1}
    \end{beamercolorbox}%

    \hskip .720\paperwidth

    \begin{beamercolorbox}[wd=.100\paperwidth,ht=3ex,dp=3ex,center]{page number in head/foot}
      \usebeamerfont{page number in head/foot}
      \iftoggle{showpagenumber}{
        \insertframenumber{} / \inserttotalframenumber{}
      }{}
      \end{beamercolorbox}%
  }%
  \vskip0pt%
}


\newcounter{SectionColorCounter}
\AtBeginSection[]
{
  \ifnum\value{SectionColorCounter}=0
    \setbeamercolor{palette primary}{fg=white, bg=google-g}
    \setbeamercolor{frametitle}{fg=white, bg=google-g}
    \setbeamercolor{structure}{fg=google-g}
    \setbeamercolor{alerted text}{fg=google-g}
  \fi
  \ifnum\value{SectionColorCounter}=1
    \setbeamercolor{palette primary}{fg=white, bg=google-b}
    \setbeamercolor{frametitle}{fg=white, bg=google-b}
    \setbeamercolor{structure}{fg=google-b}
    \setbeamercolor{alerted text}{fg=google-b}
  \fi
  \ifnum\value{SectionColorCounter}=2
    \setbeamercolor{palette primary}{fg=white, bg=google-r}
    \setbeamercolor{frametitle}{fg=white, bg=google-r}
    \setbeamercolor{structure}{fg=google-r}
    \setbeamercolor{alerted text}{fg=google-r}
  \fi
  \ifnum\value{SectionColorCounter}=3
    \setbeamercolor{palette primary}{fg=white, bg=google-y}
    \setbeamercolor{frametitle}{fg=white, bg=google-y}
    \setbeamercolor{structure}{fg=google-y}
    \setbeamercolor{alerted text}{fg=google-y}
  \fi

  \stepcounter{SectionColorCounter}
  \ifnum\value{SectionColorCounter}=4
    \setcounter{SectionColorCounter}{0}
  \fi
}

% \AtBeginSection[]
% {
%    \begin{frame}
%        \frametitle{Outline}
%        \tableofcontents[currentsection,hideothersubsections]
%    \end{frame}
% }

\newcommand{\setfooterlogo}[2]
{
  \setbeamertemplate{footline}[mine]{#1}{#2}
}

%\pgfdeclaremask{masklogo}{img/logo}
\newcommand\defaultlogo{\setfooterlogo{img/google}{}}
\defaultlogo

\setbeamersize{text margin left=10pt,text margin right=10pt}



%%%%%%%%%%%%%%%%%%%%%%%%%%%%%%%%%%%%%%%%%%%%%%%%%%%%%%%%%%%%%%%%%%%%%%%%%%%%
%% Code environnement

\definecolor{hsk-comment}   {gray}{0.5}
\colorlet{hsk-built-ins}    {google-g}
\definecolor{hsk-types}        {rgb}{0, 0.53, 0.69}
\definecolor{hsk-typevar}      {rgb}{0.82, 0.41, 0.12}
% \definecolor{hsk-typevar}      {rgb}{0.65, 0.32, 0.18}
% \definecolor{hsk-typevar}      {rgb}{0.0, 0.0, 0.8}
\definecolor{hsk-constructors} {rgb}{0.85, 0.65, 0.13}
\definecolor{hsk-operators}    {rgb}{0.84, 0.0, 0.53}
\definecolor{hsk-keywords}     {rgb}{0.0, 0.37, 0.69}
\definecolor{hsk-strings}      {rgb}{0.37, 0.53, 0.0}

\lstdefinelanguage{ColorHaskell} {
        basicstyle=\ttfamily\footnotesize,
        sensitive=true,
        morecomment=[l][\color{hsk-comment}\ttfamily\footnotesize]{--},
        morecomment=[s][\color{hsk-comment}\ttfamily\footnotesize]{\{-}{-\}},
        morestring=[b]",
        stringstyle=\color{hsk-strings},
        showstringspaces=false,
        numberstyle=none,
        showspaces=false,
        breaklines=true,
        showtabs=false,
        emph=
        {[1]
                abs,acos,acosh,all,and,any,appendFile,approxRational,asTypeOf,asin,
                asinh,atan,atan2,atanh,basicIORun,break,catch,ceiling,chr,compare,concat,concatMap,
                const,cos,cosh,curry,cycle,decodeFloat,denominator,digitToInt,divMod,drop,
                dropWhile,either,encodeFloat,enumFrom,enumFromThen,enumFromThenTo,enumFromTo,
                error,even,exp,exponent,fail,filter,flip,floatDigits,floatRadix,floatRange,floor,
                fmap,bind,ap,foldl,foldl1,foldr,foldr1,fromDouble,fromEnum,fromInt,fromInteger,fromIntegral,
                fromRational,fst,gcd,getChar,getContents,getLine,head,id,inRange,index,init,intToDigit,
                interact,ioError,isAlpha,isAlphaNum,isAscii,isControl,isDenormalized,isDigit,isHexDigit,
                isIEEE,isInfinite,isLower,isNaN,isNegativeZero,isOctDigit,isPrint,isSpace,isUpper,iterate,
                last,lcm,lex,lexDigits,lexLitChar,lines,log,logBase,lookup,map,mapM,mapM_,max,
                maxBound,maximum,maybe,min,minBound,minimum,negate,not,null,numerator,odd,
                or,ord,otherwise,pi,pred,primExitWith,print,product,properFraction,putChar,putStr,putStrLn,
                quotRem,range,rangeSize,read,readDec,readFile,readFloat,readHex,readIO,readInt,readList,readLitChar,
                readLn,readOct,readParen,readSigned,reads,readsPrec,realToFrac,recip,repeat,replicate,return,
                reverse,round,scaleFloat,scanl,scanl1,scanr,scanr1,sequence,sequence_,showChar,showInt,
                showList,showLitChar,showParen,showSigned,showString,shows,showsPrec,significand,signum,sin,
                sinh,snd,span,splitAt,sqrt,subtract,succ,sum,tail,take,takeWhile,tan,tanh,threadToIOResult,toEnum,
                toInt,toInteger,toLower,toRational,toUpper,truncate,uncurry,undefined,unlines,until,unwords,unzip,
                unzip3,userError,words,writeFile,zip,zip3,zipWith,zipWith3,listArray,doParse
        },
        emphstyle={[1]\color{hsk-built-ins}},
        emph=
        {[2]
                FilePath,IOError,Bool,Char,Double,Either,Float,IO,Integer,Int,Maybe,Ordering,Rational,Ratio,ReadS,ShowS,String, Word8,InPacket
        },
        emphstyle={[2]\color{hsk-types}},
        emph=
        {[3]
                case,class,data,deriving,do,else,if,import,in,infixl,infixr,instance,let,
                module,of,primitive,then,type,where
        },
        emphstyle={[3]\color{hsk-keywords}\textbf},
        emph=
        {[4]
                quot,rem,div,mod,elem,notElem,seq
        },
        emphstyle={[4]\color{hsk-operators}\textbf},
        emph=
        {[5]
                EQ,GT,Just,LT,Left,Nothing,Right,Show,Eq,Ord,Num,Enum,Bounded,Minutes
        },
        emphstyle={[5]\color{hsk-constructors}\textbf}
}

\lstdefinelanguage{ColorHaskellWithTypeVars} {
        basicstyle=\ttfamily\footnotesize,
        sensitive=true,
        morecomment=[l][\color{hsk-comment}\ttfamily\footnotesize]{--},
        morecomment=[s][\color{hsk-comment}\ttfamily\footnotesize]{\{-}{-\}},
        morestring=[b]",
        stringstyle=\color{hsk-strings},
        showstringspaces=false,
        numberstyle=none,
        showspaces=false,
        breaklines=true,
        showtabs=false,
        emph=
        {[1]
                abs,acos,acosh,all,and,any,appendFile,approxRational,asTypeOf,asin,
                asinh,atan,atan2,atanh,basicIORun,break,catch,ceiling,chr,compare,concat,concatMap,
                const,cos,cosh,curry,cycle,decodeFloat,denominator,digitToInt,divMod,drop,
                dropWhile,either,encodeFloat,enumFrom,enumFromThen,enumFromThenTo,enumFromTo,
                error,even,exp,exponent,fail,filter,flip,floatDigits,floatRadix,floatRange,floor,
                fmap,bind,ap,foldl,foldl1,foldr,foldr1,fromDouble,fromEnum,fromInt,fromInteger,fromIntegral,
                fromRational,fst,gcd,getChar,getContents,getLine,head,id,inRange,index,init,intToDigit,
                interact,ioError,isAlpha,isAlphaNum,isAscii,isControl,isDenormalized,isDigit,isHexDigit,
                isIEEE,isInfinite,isLower,isNaN,isNegativeZero,isOctDigit,isPrint,isSpace,isUpper,iterate,
                last,lcm,length,lex,lexDigits,lexLitChar,lines,log,logBase,lookup,map,mapM,mapM_,max,
                maxBound,maximum,maybe,min,minBound,minimum,negate,not,null,numerator,odd,
                or,ord,otherwise,pi,pred,primExitWith,print,product,properFraction,putChar,putStr,putStrLn,
                quotRem,range,rangeSize,read,readDec,readFile,readFloat,readHex,readIO,readInt,readList,readLitChar,
                readLn,readOct,readParen,readSigned,reads,readsPrec,realToFrac,recip,repeat,replicate,return,
                reverse,round,scaleFloat,scanl,scanl1,scanr,scanr1,sequence,sequence_,show,showChar,showInt,
                showList,showLitChar,showParen,showSigned,showString,shows,showsPrec,significand,signum,sin,
                sinh,snd,span,splitAt,sqrt,subtract,succ,sum,tail,take,takeWhile,tan,tanh,threadToIOResult,toEnum,
                toInt,toInteger,toLower,toRational,toUpper,truncate,uncurry,undefined,unlines,until,unwords,unzip,
                unzip3,userError,words,writeFile,zip,zip3,zipWith,zipWith3,listArray,doParse
        },
        emphstyle={[1]\color{hsk-built-ins}},
        emph=
        {[2]
                FilePath,IOError,Bool,Char,Double,Either,Float,IO,Integer,Int,Maybe,Ordering,Rational,Ratio,ReadS,ShowS,String, Word8,InPacket
        },
        emphstyle={[2]\color{hsk-types}},
        emph=
        {[3]
                case,class,data,deriving,do,else,if,import,in,infixl,infixr,instance,let,
                module,newtype,of,primitive,then,type,where
        },
        emphstyle={[3]\color{hsk-keywords}\textbf},
        emph=
        {[4]
                quot,rem,div,mod,elem,notElem,seq
        },
        emphstyle={[4]\color{hsk-operators}\textbf},
        emph=
        {[5]
                EQ,GT,Just,LT,Left,Nothing,Right,Show,Eq,Ord,Num,Enum,Bounded
        },
        emphstyle={[5]\color{hsk-constructors}\textbf},
        emph=
        {[6]
                a,b,c,m,t
        },
        emphstyle={[6]\color{hsk-typevar}\textbf}
}

\lstnewenvironment{code}[1][Haskell]
    {\lstset{
        language=ColorHaskellWithTypeVars,
        basicstyle=\footnotesize\ttfamily,
        flexiblecolumns=false,
        basewidth={0.5em,0.45em},
        escapeinside={(*}{*)},
        frame=trlb,
        frameround=tttt,
        literate={`\\}{{$\lambda$}}1
                 {`\\\\}{{\char`\\\char`\\}}1
                 {`->}{{$\rightarrow$}}2
                 {`<-}{{$\leftarrow$}}2
                 {`=>}{{$\Rightarrow$}}2
                 {`>>}{{>>}}2
                 {`>>=}{{>>=}}3
                 {`>=>}{{>=>}}3
                 {`|}{{$\mid$}}1
    }\csname lst@SetFirstLabel\endcsname}
    {\csname lst@SaveFirstLabel\endcsname\begin{flushright}{\color{lightgray!90}\vspace{-15mm}\footnotesize(#1)}\end{flushright}}

% C-like code
\lstnewenvironment{ccode}[1][C++]
    {\lstset{
        language=C,
        basicstyle=\footnotesize\ttfamily,
        flexiblecolumns=false,
        keywordstyle=\color{google-g},
        basewidth={0.5em,0.45em},
        escapeinside={(*}{*)},
        frame=trlb,
        frameround=tttt,
    }\csname lst@SetFirstLabel\endcsname}
    {\csname lst@SaveFirstLabel\endcsname\begin{flushright}{\color{lightgray!90}\vspace{-15mm}\footnotesize(#1)}\end{flushright}}

% Python code
\lstnewenvironment{pycode}[1][Python]
    {\lstset{
        language=Python,
        basicstyle=\footnotesize\ttfamily,
        flexiblecolumns=false,
        keywordstyle=\color{google-g},
        basewidth={0.5em,0.45em},
        escapeinside={(*}{*)},
        frame=trlb,
        frameround=tttt,
    }\csname lst@SetFirstLabel\endcsname}
    {\csname lst@SaveFirstLabel\endcsname\begin{flushright}{\color{lightgray!90}\vspace{-15mm}\footnotesize(#1)}\end{flushright}}

\def\inlinecode{\lstinline[language=ColorHaskell,
      basicstyle=\footnotesize\ttfamily,
      flexiblecolumns=false,
      basewidth={0.5em,0.45em},
      literate={`\\}{{$\lambda$}}1
               {`\\\\}{{\char`\\\char`\\}}1
               {`->}{{$\rightarrow$}}2
               {`<-}{{$\leftarrow$}}2
               {`=>}{{$\Rightarrow$}}2
               {`>>}{{>>}}2
               {`>>=}{{>>=}}3
               {`>=>}{{>=>}}3
               {`|}{{$\mid$}}1]
  }

\newcommand<>{\ic}{\inlinecode}

\def\inlinecodewithtypevars{\lstinline[language=ColorHaskellWithTypeVars,
      basicstyle=\footnotesize\ttfamily,
      flexiblecolumns=false,
      basewidth={0.5em,0.45em},
      literate={`\\}{{$\lambda$}}1
               {`\\\\}{{\char`\\\char`\\}}1
               {`->}{{$\rightarrow$}}2
               {`<-}{{$\leftarrow$}}2
               {`=>}{{$\Rightarrow$}}2
               {`>>}{{>>}}2
               {`>>=}{{>>=}}3
               {`>=>}{{>=>}}3
               {`|}{{$\mid$}}1]
  }

\newcommand<>{\icwtv}{\inlinecodewithtypevars}


\newcommand<>{\type}[1]{\textcolor{hsk-types}{#1}}
\newcommand<>{\tvar}[1]{\textcolor{hsk-typevar}{#1}}
\newcommand<>{\cons}[1]{\textcolor{hsk-constructors}{#1}}
\newcommand<>{\bi}[1]{\textcolor{hsk-built-ins}{#1}}
\newcommand<>{\str}[1]{\textcolor{hsk-strings}{#1}}

\newcommand<>{\ict}[1]{\type{\ic{#1}}}
\newcommand<>{\icv}[1]{\tvar{\ic{#1}}}
\newcommand<>{\icc}[1]{\cons{\ic{#1}}}
\newcommand<>{\icbi}[1]{\bi{\ic{#1}}}



%%%%%%%%%%%%%%%%%%%%%%%%%%%%%%%%%%%%%%%%%%%%%%%%%%%%%%%%%%%%%%%%%%%%%%%%%%%%
%% Fonts

\setbeamerfont{frametitle}{size=\small}
\setbeamerfont{structure}{series=\bfseries}



%%%%%%%%%%%%%%%%%%%%%%%%%%%%%%%%%%%%%%%%%%%%%%%%%%%%%%%%%%%%%%%%%%%%%%%%%%%%
%% Settings

\setbeamertemplate{navigation symbols}{}

\bibliographystyle{apalike}

\mode<all>


\renewcommand{\(}[1]{\begin{columns}[#1]}
\renewcommand{\)}{\end{columns}}
\newcommand{\<}[1]{\begin{column}{#1}}
\renewcommand{\>}{\end{column}}

\newcommand{\Split}[3][.5]{%
\({c}%
  \<{{#1}\linewidth}%
  \begin{minipage}[c][.8\textheight]{\linewidth}%
  \begin{center}%
    {#2}%
  \end{center}%
  \end{minipage}%
  \>%
  \<{\linewidth-{#1}\linewidth}%
  \begin{minipage}[c][.8\textheight]{\linewidth}%
  \begin{center}%
    {#3}%
  \end{center}%
  \end{minipage}
  \>%
\)%
}

\newcommand<>{\Image}[3][]{%
\IfFileExists{#3.png}{%
\includegraphics#4[width=\linewidth,height=#2,keepaspectratio,#1]{#3}%
}{%
\IfFileExists{#3.jpg}{%
\includegraphics#4[width=\linewidth,height=#2,keepaspectratio,#1]{#3}%
}{%
\includegraphics#4[width=\linewidth,height=#2,keepaspectratio,#1]{img/placeholder}%
}}}

\newenvironment{TopAlign}[2][1]{\begin{minipage}[t][#2\textheight]{#1\linewidth}}{\end{minipage}}

\newcommand{\pc}[1]{{\footnotesize\texttt{#1}}}
\newcommand<>{\opc}[1]{\only#2{\pc{#1}}}
\newcommand<>{\upc}[1]{\uncover#2{\pc{#1}}}
\newcommand<>{\oic}[1]{\only#2{\ic{#1}}}
\newcommand<>{\uic}[1]{\uncover#2{\ic{#1}}}

\def\shruggie{\texttt{\raisebox{0.75em}{\char`\_}\char`\\\char`\_\kern-0.5ex(\kern-0.25ex\raisebox{0.25ex}{\rotatebox{45}{\raisebox{-.75ex}"\kern-1.5ex\rotatebox{-90})}}\kern-0.5ex)\kern-0.5ex\char`\_/\raisebox{0.75em}{\char`\_}}}


%%%%%%%%%%%%%%%%%%%%%%%%%%%%%%%%%%%%%%%%%%%%%%%%%%%%%%%%%%%%%%%%%%%%%%%%%%%%
%% Document

\begin{document}


% Set-up: have on screen while people are gathering
\togglefalse{showpagenumber}
\begin{frame}[fragile]
  \frametitle{Before we begin}
  \begin{code}[setup]
cd 00_setup && make
# Windows users: see setup.md
  \end{code}
\end{frame}


% Title frame

\begin{frame}[fragile]
  \titlepage
\end{frame}
\toggletrue{showpagenumber}
\setcounter{framenumber}{0}


\section{Intro}

\begin{frame}
  \frametitle{Goals}
  \Split{%
    \begin{itemize}
      \item 101: basic skills
        \begin{itemize}
          \item Reading function types
          \item Pattern matching
          \item Data structures
        \end{itemize}
    \end{itemize}
    \begin{itemize}
      \item 102: first project
        \begin{itemize}
          \item Genericity
          \item IO and do notation
	  \item Build a game! (demo)
        \end{itemize}
    \end{itemize}
  }{%
    \Image{3cm}{img/success}
  }%
\end{frame}

\begin{frame}
  \frametitle{Roadmap}
  \begin{TopAlign}{.52}
    \begin{itemize}
      \item 101 Recap
      \item Remaining obstacles
      \item Typeclasses overview
      \item \alt<2>{Typeclasses}{Common examples}
      \item \alt<2>{Typeclasses}{Advanced syntax}
      \item<2> Typeclasses\ldots
    \end{itemize}
  \end{TopAlign}
\end{frame}


\section{Recap}

\begin{frame}[fragile]
  \frametitle{Curried functions, partial application}
  \begin{onlyenv}<1>
    \begin{code}
(*\\*)
 f      :: Int ->   Int -> [Int]
(*\\*)
(*\\*)
(*\\*)
    \end{code}
  \end{onlyenv}
  \begin{onlyenv}<2>
    \begin{code}
 f      :: Int -> ( Int -> [Int] )
 f      :: Int ->   Int -> [Int]
(*\\*)
(*\\*)
(*\\*)
    \end{code}
  \end{onlyenv}
  \begin{onlyenv}<3>
    \begin{code}
 f      :: Int -> ( Int -> [Int] )
 f      :: Int ->   Int -> [Int]
 f 1    ::          Int -> [Int]
(*\\*)
(*\\*)
    \end{code}
  \end{onlyenv}
  \begin{onlyenv}<4>
    \begin{code}
 f      :: Int -> ( Int -> [Int] )
 f      :: Int ->   Int -> [Int]
 f 1    ::          Int -> [Int]
(*\\*)
(f 1) 2 ::                 [Int]
    \end{code}
  \end{onlyenv}
  \begin{onlyenv}<5>
    \begin{code}
 f      :: Int -> ( Int -> [Int] )
 f      :: Int ->   Int -> [Int]
 f 1    ::          Int -> [Int]
 f 1  2 ::                 [Int]
(f 1) 2 ::                 [Int]
    \end{code}
  \end{onlyenv}
\end{frame}

\begin{frame}[fragile]
  \frametitle{Type constructors}
  \begin{center}
    \begin{uncoverenv}<1->
      \begin{code}
-- enum
data (*\type{Bool}*)  = (*\cons{False}*) | (*\cons{True}*)
data (*\type{Color}*) = (*\cons{Red}*) | (*\cons{Green}*) | (*\cons{Blue}*)
      \end{code}
    \end{uncoverenv}
    \begin{uncoverenv}<2->
      \begin{code}
-- struct
data (*\type{Point}*) = (*\cons{Point}*) { x :: (*\type{Double}*), y :: (*\type{Double}*) }
      \end{code}
    \end{uncoverenv}
    \begin{uncoverenv}<3->
      \begin{onlyenv}<-3>
        \begin{code}
-- a bit more interesting
data (*\type{Minutes}*) = (*\cons{Minutes}*) (*\type{Int}*)
data (*\type{Maybe}*) (*\tvar{a}*) = (*\cons{Nothing}*) | (*\cons{Just}*) (*\tvar{a}*)
data (*\type{List}*) (*\tvar{a}*)  = (*\cons{Nil}*) | (*\cons{Cell}*) (*\tvar{a}*) ((*\cons{List}*) (*\tvar{a}*))
        \end{code}
      \end{onlyenv}
      \begin{onlyenv}<4>
        \begin{code}
-- a bit more interesting
data (*\type{Minutes}*) = (*\cons{Minutes}*) (*\type{Int}*)
data (*\type{Maybe}*) (*\tvar{a}*) = (*\cons{Nothing}*) | (*\cons{Just}*) (*\tvar{a}*)
data [(*\tvar{a}*)]     = [] | ((*\tvar{a}*):[(*\tvar{a}*)])
        \end{code}
      \end{onlyenv}
    \end{uncoverenv}
  \end{center}
\end{frame}

\begin{frame}[fragile]
  \frametitle{``Deconstructors'' and pattern matching}
  \begin{center}
    \begin{uncoverenv}<1->
      \begin{code}
not :: (*\type{Bool}*) -> (*\type{Bool}*)
not (*\cons{True}*)  = (*\cons{False}*)
not (*\cons{False}*) = (*\cons{True}*)
      \end{code}
    \end{uncoverenv}
    \begin{uncoverenv}<2->
      \begin{code}
magnitude :: (*\type{Point}*) -> (*\type{Double}*)
magnitude ((*\cons{Point}*) x y) = sqrt $ x^2 + y^2
    \end{code}%$
    \end{uncoverenv}
    \begin{uncoverenv}<3->
      \begin{code}
length :: [(*\tvar{a}*)] -> (*\type{Int}*)
length []     = 0
length (_:xs) = 1 + length xs
      \end{code}
    \end{uncoverenv}
  \end{center}
\end{frame}

\begin{frame}
  \frametitle{We know}
  \begin{center}
    \begin{minipage}[c]{.8\linewidth}
      \begin{itemize}
        \item how to read function types
        \item how to declare new types
        \item how to do pattern matching
      \end{itemize}
    \end{minipage}
  \end{center}
\end{frame}


\section{Obstacles}

\begin{frame}[fragile]
  \frametitle{Our types are limited}
  \begin{TopAlign}{.5}
    \begin{flushleft}
      \footnotesize\texttt{\$ ghci}                                                ~\\
      \uncover<1->  {\footnotesize\texttt{> \uncover<2-> {\ic{data\ }\ict{Color}\ic{\ =\ }\icc{Red}\ic{\ |\ }\icc{Green}\ic{\ |\ }\icc{Blue}}}} ~\\
      \uncover<3->  {\footnotesize\texttt{> \uncover<4-> {\ic{let mycolor =\ }\icc{Red}}}}        ~\\
      \begin{onlyenv}<5-7>
        \uncover<5-7> {\footnotesize\texttt{> \uncover<6-7>{\ic{mycolor}}}}                         ~\\
        \uncover<7>   {\footnotesize\texttt{~~~~No instance for (\ict{Show Color})}}                ~\\
        \uncover<7>   {\footnotesize\texttt{~~~~arising from a use of `print'}}                     ~\\
      \end{onlyenv}
      \begin{onlyenv}<8-10>
        \uncover<8-10>{\footnotesize\texttt{> \uncover<9-10>{\ic{mycolor ==\ }\icc{Red}}}}          ~\\
        \uncover<10>  {\footnotesize\texttt{~~~~No instance for (\ict{Eq Color})}}                  ~\\
        \uncover<10>  {\footnotesize\texttt{~~~~arising from a use of `(==)'}}                      ~\\
      \end{onlyenv}
    \end{flushleft}
  \end{TopAlign}
\end{frame}

\begin{frame}[fragile]
  \frametitle{Type parameters constraints?}
  \begin{code}
sumI :: [(*\type{Int}*)] -> (*\type{Int}*)
sumI = foldl (+) 0
(*\\*)
sumD :: [(*\type{Double}*)] -> (*\type{Double}*)
sumD = foldl (+) 0
  \end{code}
  \pause~\\
  \begin{code}
sum :: [(*\tvar{a}*)] -> (*\tvar{a}*)
sum = foldl (+) 0
  \end{code}
  \pause
  \begin{flushleft}
    \footnotesize
    \texttt{~~No instance for (\ict{Num} \icv{a})}\\
    \texttt{~~arising from a use of `(+)'}
  \end{flushleft}
\end{frame}

\begin{frame}[fragile]
  \frametitle{Cascading context}
  \begin{code}
getUser        :: (*\type{Id}*)   -> (*\type{Maybe}*) (*\type{User}*)
getNextOfKin   :: (*\type{User}*) -> (*\type{Maybe}*) (*\type{Id}*)
getPhoneNumber :: (*\type{User}*) -> (*\type{Maybe}*) (*\type{PhoneNumber}*)
  \end{code}
  \begin{onlyenv}<2>
    \begin{code}
getNextOfKinPhoneNumber :: (*\type{Id}*) -> (*\type{Maybe}*) (*\type{PhoneNumber}*)
(*\\*)
(*\\*)
(*\\*)
(*\\*)
(*\\*)
(*\\*)
(*\\*)
    \end{code}
  \end{onlyenv}
  \begin{onlyenv}<3>
    \begin{code}
getNextOfKinPhoneNumber :: (*\type{Id}*) -> (*\type{Maybe}*) (*\type{PhoneNumber}*)
getNextOfKinPhoneNumber userId =
(*\\*)
(*\\*)
(*\\*)
(*\\*)
(*\\*)
(*\\*)
    \end{code}
  \end{onlyenv}
  \begin{onlyenv}<4>
    \begin{code}
getNextOfKinPhoneNumber :: (*\type{Id}*) -> (*\type{Maybe}*) (*\type{PhoneNumber}*)
getNextOfKinPhoneNumber userId = case getUser userId of
  Nothing   -> Nothing
  Just user ->
(*\\*)
(*\\*)
(*\\*)
(*\\*)
    \end{code}
  \end{onlyenv}
  \begin{onlyenv}<5>
    \begin{code}
getNextOfKinPhoneNumber :: (*\type{Id}*) -> (*\type{Maybe}*) (*\type{PhoneNumber}*)
getNextOfKinPhoneNumber userId = case getUser userId of
  Nothing   -> Nothing
  Just user -> case getNextOfKin user of
    Nothing          -> Nothing
    Just nextOfKinId ->
(*\\*)
(*\\*)
    \end{code}
  \end{onlyenv}
  \begin{onlyenv}<6>
    \begin{code}
getNextOfKinPhoneNumber :: (*\type{Id}*) -> (*\type{Maybe}*) (*\type{PhoneNumber}*)
getNextOfKinPhoneNumber userId = case getUser userId of
  Nothing   -> Nothing
  Just user -> case getNextOfKin user of
    Nothing          -> Nothing
    Just nextOfKinId -> case getUser nextOfKinId of
      Nothing        -> Nothing
      Just nextOfKin ->
    \end{code}
  \end{onlyenv}
  \begin{onlyenv}<7>
    \begin{code}
getNextOfKinPhoneNumber :: (*\type{Id}*) -> (*\type{Maybe}*) (*\type{PhoneNumber}*)
getNextOfKinPhoneNumber userId = case getUser userId of
  Nothing   -> Nothing
  Just user -> case getNextOfKin user of
    Nothing          -> Nothing
    Just nextOfKinId -> case getUser nextOfKinId of
      Nothing        -> Nothing
      Just nextOfKin -> getPhoneNumber nextOfKin
    \end{code}
  \end{onlyenv}
\end{frame}

\begin{frame}
  \frametitle{IO}
  \begin{TopAlign}{.1}
    \begin{minipage}[c][.3\textheight]{\linewidth}%
      \begin{center}
        \begin{itemize}
          \item<1-> Can't apply regular functions to IO values
          \item<6-> Can't get values out of IO
          \item<7-> Can't pattern match on IO
        \end{itemize}
      \end{center}
    \end{minipage}
  \end{TopAlign}
  \begin{minipage}[c][.5\textheight]{\linewidth}%
    \begin{center}
      \begin{onlyenv}<2-5>
        \begin{flushleft}
          \footnotesize\texttt{\$ ghci}                                                      ~\\
          \uncover<2->{\footnotesize\texttt{> \uncover<3-> {\ic{let name = getLine --}\ict{\ IO String}}}} ~\\
          \uncover<3->{\footnotesize\texttt{> \uncover<4-> {\ic{putStr \$ "Hello " ++ name ++ "!"}}}} ~\\
          \uncover<5->{\footnotesize\texttt{~~~~Expected type: `\ict{String}'}}          ~\\
          \uncover<5->{\footnotesize\texttt{~~~~~~Actual type: `\ict{IO String}'}}       ~\\
          \uncover<5->{\footnotesize\texttt{~~~~In second argument of (++)}}             ~\\
        \end{flushleft}
      \end{onlyenv}
      \begin{center}
        \only<8>{IO's implementation\\details are hidden}
      \end{center}
    \end{center}
  \end{minipage}
\end{frame}

\begin{frame}
  \frametitle{We don't know}
  \begin{center}
    \begin{minipage}[c]{.8\linewidth}
      \begin{itemize}
        \item how to extend our data types
        \item how to express type constraints
        \item how to chain contextual functions
        \item how to use IO
      \end{itemize}
    \end{minipage}
  \end{center}
\end{frame}


\section{Typeclasses}

\begin{frame}[fragile]
  \frametitle{Typeclasses}
  \begin{code}[typeclass]
class (*\type{Show}*) (*\tvar{a}*) where
  show :: (*\tvar{a}*) -> (*\type{String}*)
  \end{code}
  \begin{uncoverenv}<2->
    \begin{code}[type]
data (*\type{Color}*) = (*\cons{Red}*) | (*\cons{Green}*) | (*\cons{Blue}*)
    \end{code}
  \end{uncoverenv}
  \begin{uncoverenv}<3->
    \begin{code}[instance]
instance (*\type{Show}*) (*\type{Color}*) where
  show (*\cons{Red}*)   = "Red"
  show (*\cons{Green}*) = "Green"
  show (*\cons{Blue}*)  = "Blue"
    \end{code}
  \end{uncoverenv}
\end{frame}

\begin{frame}[fragile]
  \frametitle{Constraints}
  \begin{code}[Show]
show :: (*\type{Show}*) (*\tvar{a}*) => (*\tvar{a}*) -> (*\type{String}*)
  \end{code}
  \begin{uncoverenv}<2->
    \begin{code}[Num]
sum  :: (*\type{Num}*) (*\tvar{a}*)  => [(*\tvar{a}*)] -> (*\tvar{a}*)
    \end{code}
  \end{uncoverenv}
  \begin{uncoverenv}<3->
    \begin{code}[Eq]
(==) :: (*\type{Eq}*) (*\tvar{a}*)   => (*\tvar{a}*) -> (*\tvar{a}*) -> (*\type{Bool}*)
    \end{code}
  \end{uncoverenv}
\end{frame}

\begin{frame}[fragile]
  \frametitle{Constraints in instance declarations}
  \begin{onlyenv}<1>
    \begin{code}
instance (*\type{Show}*) ((*\type{Maybe}*) (*\tvar{a}*)) where
(*\\*)
    \end{code}
  \end{onlyenv}
  \begin{onlyenv}<2>
    \begin{code}
instance (*\type{Show}*) ((*\type{Maybe}*) (*\tvar{a}*)) where
  show (*\cons{Nothing}*)  = "Nothing"
    \end{code}
  \end{onlyenv}
  \begin{onlyenv}<3>
    \begin{code}
instance (*\type{Show}*) ((*\type{Maybe}*) (*\tvar{a}*)) where
  show (*\cons{Nothing}*)  = "Nothing"
  show ((*\cons{Just}*) x) = "Just " ++ show x
    \end{code}
  \end{onlyenv}
  \begin{onlyenv}<4>
    \begin{code}
instance (*\type{Show}*) (*\tvar{a}*) => (*\type{Show}*) ((*\type{Maybe}*) (*\tvar{a}*)) where
  show (*\cons{Nothing}*)  = "Nothing"
  show ((*\cons{Just}*) x) = "Just " ++ show x
    \end{code}
  \end{onlyenv}
\end{frame}

\begin{frame}[fragile]
  \frametitle{Quick tour}
  \frametitle<9->{Quick boring tour...}
  \begin{center}
    \begin{minipage}[t][.2\textheight]{.8\linewidth}
      \begin{center}
        \begin{tabular}{ c c c c c c }
          \alt<2-5>{Show}{\color{lightgray}Show}&
          \alt<4-5>{Read}{\color{lightgray}Read}&
          \alt<6-9>{Eq}{\color{lightgray}Eq}&
          \alt<10-11>{Ord}{\color{lightgray}Ord}&
          \alt<12>{Bounded}{\color{lightgray}Bounded}&
          \alt<13>{Enum}{\color{lightgray}Enum}
        \end{tabular}
        \hrule
      \end{center}
    \end{minipage}
    \begin{minipage}[t][.6\textheight]{.8\linewidth}
      \begin{onlyenv}<2-9>
        \begin{code}[def]
data (*\type{Color}*) = (*\cons{Red}*) | (*\cons{Green}*) | (*\cons{Blue}*)
        \end{code}
      \end{onlyenv}
      \begin{onlyenv}<2-5>
        \begin{flushleft}\footnotesize
          ~\\\uncover<2->{\texttt{> show \cons{Blue}}}
          ~\\\uncover<3->{\texttt{\str{"Blue"}}}
          ~\\\uncover<4->{\texttt{> read \str{"Green"} :: \type{Color}}}
          ~\\\uncover<5->{\texttt{\cons{Green}}}
        \end{flushleft}
      \end{onlyenv}
      \begin{onlyenv}<6>
        \begin{code}
class (*\type{Eq}*) (*\tvar{a}*) where
  (==) :: (*\tvar{a}*) -> (*\tvar{a}*) -> (*\type{Bool}*)
  (*\\*)
  (/=) :: (*\tvar{a}*) -> (*\tvar{a}*) -> (*\type{Bool}*)
  (*\\*)
        \end{code}
      \end{onlyenv}
      \begin{onlyenv}<7>
        \begin{code}
class (*\type{Eq}*) (*\tvar{a}*) where
  (==) :: (*\tvar{a}*) -> (*\tvar{a}*) -> (*\type{Bool}*)
  (==) a b = not $ a /= b
  (/=) :: (*\tvar{a}*) -> (*\tvar{a}*) -> (*\type{Bool}*)
  (/=) a b = not $ a == b
        \end{code}
      \end{onlyenv}
      \begin{onlyenv}<8>
        \begin{code}
class (*\type{Eq}*) (*\tvar{a}*) where
  (==) :: (*\tvar{a}*) -> (*\tvar{a}*) -> (*\type{Bool}*)
  a == b = not $ a /= b
  (/=) :: (*\tvar{a}*) -> (*\tvar{a}*) -> (*\type{Bool}*)
  a /= b = not $ a == b
        \end{code}
      \end{onlyenv}
      \begin{onlyenv}<9>
        \begin{code}
instance (*\type{Eq}*) (*\type{Color}*) where
  (*\cons{Red}*)   == (*\cons{Red}*)   = (*\cons{True}*)
  (*\cons{Green}*) == (*\cons{Green}*) = (*\cons{True}*)
  (*\cons{Blue}*)  == (*\cons{Blue}*)  = (*\cons{True}*)
  _     == _     = (*\cons{False}*)
        \end{code}
      \end{onlyenv}
      \begin{onlyenv}<10>
        \begin{code}
class (*\type{Eq}*) (*\tvar{a}*) => (*\type{Ord}*) (*\tvar{a}*) where
  compare :: (*\tvar{a}*) -> (*\tvar{a}*) -> (*\type{Ordering}*)
  (<=)    :: (*\tvar{a}*) -> (*\tvar{a}*) -> (*\type{Bool}*)
  (>=)    :: (*\tvar{a}*) -> (*\tvar{a}*) -> (*\type{Bool}*)
  (<)     :: (*\tvar{a}*) -> (*\tvar{a}*) -> (*\type{Bool}*)
  (>)     :: (*\tvar{a}*) -> (*\tvar{a}*) -> (*\type{Bool}*)
  max     :: (*\tvar{a}*) -> (*\tvar{a}*) -> (*\tvar{a}*)
  min     :: (*\tvar{a}*) -> (*\tvar{a}*) -> (*\tvar{a}*)
        \end{code}
      \end{onlyenv}
      \begin{onlyenv}<11>
        \begin{code}
class (*\type{Eq}*) (*\tvar{a}*) => (*\type{Ord}*) (*\tvar{a}*) where
  compare :: (*\tvar{a}*) -> (*\tvar{a}*) -> (*\type{Ordering}*)
  (<=)    :: (*\tvar{a}*) -> (*\tvar{a}*) -> (*\type{Bool}*)
        \end{code}
      \end{onlyenv}
      \begin{onlyenv}<12>
        \begin{code}
class (*\type{Bounded}*) (*\tvar{a}*) where
  minBound :: (*\tvar{a}*)
  maxBound :: (*\tvar{a}*)
        \end{code}
      \end{onlyenv}
      \begin{onlyenv}<13>
        \begin{code}
class (*\type{Enum}*) (*\tvar{a}*) where
  succ           :: (*\tvar{a}*) -> (*\tvar{a}*)
  pred           :: (*\tvar{a}*) -> (*\tvar{a}*)
  toEnum         :: (*\type{Int}*) -> (*\tvar{a}*)
  fromEnum       :: (*\tvar{a}*) -> (*\type{Int}*)
  enumFrom       :: (*\tvar{a}*) -> [(*\tvar{a}*)]
  enumFromThen   :: (*\tvar{a}*) -> (*\tvar{a}*) -> [(*\tvar{a}*)]
  enumFromTo     :: (*\tvar{a}*) -> (*\tvar{a}*) -> [(*\tvar{a}*)]
  enumFromThenTo :: (*\tvar{a}*) -> (*\tvar{a}*) -> (*\tvar{a}*) -> [(*\tvar{a}*)]
        \end{code}
      \end{onlyenv}
    \end{minipage}
  \end{center}
\end{frame}

\begin{frame}[fragile]
  \frametitle{Deriving}
  \begin{center}%
    \begin{onlyenv}<1>
      \begin{code}
data (*\type{Color}*) = (*\cons{Red}*) | (*\cons{Green}*) | (*\cons{Blue}*)
(*\\*)
(*\\*)
(*\\*)
(*\\*)
(*\\*)
(*\\*)
(*\\*)
      \end{code}
    \end{onlyenv}
    \begin{onlyenv}<2>
      \begin{code}
data (*\type{Color}*) = (*\cons{Red}*) | (*\cons{Green}*) | (*\cons{Blue}*)
  deriving ( (*\type{Show}*)
           , (*\type{Read}*)
           , (*\type{Eq}*)
           , (*\type{Ord}*)
           , (*\type{Bounded}*)
           , (*\type{Enum}*)
           )
      \end{code}
    \end{onlyenv}
  \end{center}
\end{frame}


\section{Codelab - part 1}

\begin{frame}
  \frametitle{Codelab :: Section 1}
  \begin{description}
    \item[directory]: \ic{01_mastermind/}
    \item[change]: \ic{src/Color.hs} \\
      (replace \ic{codelab} with implementation)
    \item[check with]: \ic{make ARGS="check 1"}
  \end{description}
\end{frame}

\begin{frame}
  \frametitle{Codelab :: Section 2}
  \begin{description}
    \item[directory]: \ic{01_mastermind/}
    \item[change]: \ic{src/ColorMap.hs} \\
      (replace \ic{codelab} with implementation)
    \item[check with]: \ic{make ARGS="check 2"}
  \end{description}
\end{frame}

\begin{frame}
  \frametitle{Contexts / wrappers}
  \begin{minipage}[c]{\linewidth}
    \begin{center}
      \begin{tabular}{ r c r }
        \uncover<1->{A          &~~~~~$\rightarrow$~~~~~& value} \\\\
        \uncover<2->{Maybe A    &~~~~~$\rightarrow$~~~~~& optional value} \\
        \uncover<3->{List A     &~~~~~$\rightarrow$~~~~~& repeated value} \\
        \uncover<4->{IO A       &~~~~~$\rightarrow$~~~~~& impure value}   \\\\
        \uncover<5->{C~~A       &~~~~~$\rightarrow$~~~~~& "contextual" value}   \\
      \end{tabular}
    \end{center}
  \end{minipage}
\end{frame}

\begin{frame}
  \frametitle{Similarities}
  \begin{itemize}
    \uncover<1->{\item Wrapping is trivial}
    \uncover<5->{\item Unwrapping is non-trivial / destructive / impossible}
    \uncover<9->{\item What we want: to apply functions \textbf{in the context}}
  \end{itemize}
  \begin{flushleft}
    \begin{onlyenv}<2-4>
      \uncover<2-4>{\footnotesize\texttt{~~~~\ic{wrap x = [x]}}\\}
      \uncover<3-4>{\footnotesize\texttt{~~~~\ic{wrap x = Just x}}\\}
      \uncover<4>{\footnotesize\texttt{~~~~\ic{wrap x = pure x}}\\}
    \end{onlyenv}
    \begin{onlyenv}<6-8>
      \uncover<6->{\footnotesize\texttt{~~~~\ic{unwrap Nothing = ???}}\\}
      \uncover<7->{\footnotesize\texttt{~~~~\ic{unwrap [1,2,3] = ???}}\\}
      \uncover<8->{\footnotesize\texttt{~~~~\ic{unwrap getLine =} {\color{red}NOPE}}}\\
    \end{onlyenv}
  \end{flushleft}
\end{frame}

\begin{frame}
  \frametitle{Context functions}
  \begin{minipage}[c][.2\textheight]{\linewidth}%
    \begin{center}
      Three standard functions to deal with contexts
    \end{center}
  \end{minipage}
  \begin{minipage}[t][.5\textheight]{\linewidth}%
    \begin{flushleft}
      \upc<2->{~\\~~fmap :: ~~\icwtv{(a -> b) -> c a -> c b}\\}
      \upc<3->{~\\~~ap~~ :: \icwtv{c (a -> b) -> c a -> c b}\\}
      \upc<4->{~\\~~bind :: \icwtv{(a -> c b) -> c a -> c b}\\}
    \end{flushleft}
  \end{minipage}
\end{frame}

\begin{frame}[fragile]
  \frametitle{fmap}
  \begin{itemize}
    \uncover<1->{\item fmap is like map\ldots}
    \uncover<3->{\item but generalised to all contexts}
    \uncover<13->{\item and it also has an operator version}
  \end{itemize}
  \begin{minipage}[t][.4\textheight]{\linewidth}
    \begin{onlyenv}<2>
      \begin{code}
 map :: (a -> b) ->   [a] ->   [b]
(*\\*)
      \end{code}
    \end{onlyenv}
    \begin{onlyenv}<3>
      \begin{code}
 map :: (a -> b) ->   [a] ->   [b]
fmap :: (a -> b) -> c  a  -> c  b
      \end{code}
    \end{onlyenv}
    \begin{onlyenv}<4>
      \begin{code}
fmap  show  [1, 2, 3] =
(*\\*)
(*\\*)
(*\\*)
      \end{code}
    \end{onlyenv}
    \begin{onlyenv}<5>
      \begin{code}
fmap  show  [1, 2, 3] = ["1", "2", "3"]
(*\\*)
(*\\*)
(*\\*)
      \end{code}
    \end{onlyenv}
    \begin{onlyenv}<6>
      \begin{code}
fmap  show  [1, 2, 3] = ["1", "2", "3"]
fmap  show  (Just 42) =
(*\\*)
(*\\*)
      \end{code}
    \end{onlyenv}
    \begin{onlyenv}<7>
      \begin{code}
fmap  show  [1, 2, 3] = ["1", "2", "3"]
fmap  show  (Just 42) = Just "42"
(*\\*)
(*\\*)
      \end{code}
    \end{onlyenv}
    \begin{onlyenv}<8>
      \begin{code}
fmap  show  [1, 2, 3] = ["1", "2", "3"]
fmap  show  (Just 42) = Just "42"
fmap  show   Nothing  =
(*\\*)
      \end{code}
    \end{onlyenv}
    \begin{onlyenv}<9>
      \begin{code}
fmap  show  [1, 2, 3] = ["1", "2", "3"]
fmap  show  (Just 42) = Just "42"
fmap  show   Nothing  = Nothing
(*\\*)
      \end{code}
    \end{onlyenv}
    \begin{onlyenv}<10>
      \begin{code}
fmap  show  [1, 2, 3] = ["1", "2", "3"]
fmap  show  (Just 42) = Just "42"
fmap  show   Nothing  = Nothing
fmap length  getLine  =
      \end{code}
    \end{onlyenv}
    \begin{onlyenv}<11>
      \begin{code}
fmap  show  [1, 2, 3] = ["1", "2", "3"]
fmap  show  (Just 42) = Just "42"
fmap  show   Nothing  = Nothing
fmap length  getLine  = (*\shruggie*)
      \end{code}
    \end{onlyenv}
    \begin{onlyenv}<12>
      \begin{code}
fmap  show  [1, 2, 3] = ["1", "2", "3"]
fmap  show  (Just 42) = Just "42"
fmap  show   Nothing  = Nothing
fmap length  getLine  = (*\shruggie*)   --  (*\type{IO Int}*)
      \end{code}
    \end{onlyenv}
    \begin{onlyenv}<13>
      \begin{code}
show    <$> [1, 2, 3] = ["1", "2", "3"]
show    <$> (Just 42) = Just "42"
show    <$>  Nothing  = Nothing
length  <$>  getLine  = (*\shruggie*)   --  (*\type{IO Int}*)
      \end{code}
    \end{onlyenv}
  \end{minipage}
\end{frame}

\begin{frame}[fragile]
  \frametitle{When to use fmap}
  \begin{itemize}
    \item We have one or more \textbf{``wrapped''} values
    \item Want to apply a function \textbf{uniformly} for all values
  \end{itemize}
  \begin{minipage}[t][.4\textheight]{\linewidth}
    \begin{center}
      \begin{tabular}{ r c r }
        \uncover<1->{                   & \icwtv{f :: a -> b} &                   } \\
        \uncover<1->{\icwtv{Maybe a}    &  $\longrightarrow$  & \icwtv{Maybe b}   } \\
        \uncover<2->{\icwtv{Minutes a}  &  $\longrightarrow$  & \icwtv{Minutes b} } \\
        \uncover<1->{\icwtv{[a]}        &  $\longrightarrow$  & \icwtv{[b]}       } \\
        \uncover<2->{\icwtv{BinTree a}  &  $\longrightarrow$  & \icwtv{BinTree b} } \\
        \uncover<3->{\icwtv{Either c a} &  $\longrightarrow$  & \icwtv{Either c b}} \\
      \end{tabular}
    \end{center}
  \end{minipage}
\end{frame}

\begin{frame}[fragile]
  \frametitle{``Lifting'' a function}
  \begin{code}
fmap :: (a -> b) ->  f a -> f b
fmap :: (a -> b) -> (f a -> f b)
  \end{code}
  \begin{minipage}[t][.4\textheight]{\linewidth}
  \begin{center}
  \tikzstyle{sqr} = [draw=black,
    minimum width=0.5cm,
    minimum height=0.5cm,
    node distance=1.3cm]
  \begin{tikzpicture}
    \node (f) {\icwtv{g :: a -> b}};
    \node[below of=f, node distance=1cm] (ff) {};
    \node[below of=ff, node distance=1cm] (fff) {};
    \node[left of=fff, node distance=1cm] (fffl) {};
    \node[right of=fff, node distance=1cm] (fffr) {};
    \path[draw, -latex] (fffl.west) -- node[above] {\icwtv{fmap g}} (fffr.east);
    \node[sqr, left of=ff, node distance=3cm] (a1) {\icv{a}};
    \node[sqr, below left of=a1] (a2) {\icv{a}};
    \node[sqr, below right of=a1] (a3) {\icv{a}};
    \node[sqr, below left of=a2] (a4) {\icv{a}};
    \node[sqr, below right of=a2] (a5) {\icv{a}};
    \path[draw] (a1.south) -- (a2.north);
    \path[draw] (a1.south) -- (a3.north);
    \path[draw] (a2.south) -- (a4.north);
    \path[draw] (a2.south) -- (a5.north);
    \node[sqr, right of=ff, node distance=3cm] (b1) {\icv{b}};
    \node[sqr, below left of=b1] (b2) {\icv{b}};
    \node[sqr, below right of=b1] (b3) {\icv{b}};
    \node[sqr, below left of=b2] (b4) {\icv{b}};
    \node[sqr, below right of=b2] (b5) {\icv{b}};
    \path[draw] (b1.south) -- (b2.north);
    \path[draw] (b1.south) -- (b3.north);
    \path[draw] (b2.south) -- (b4.north);
    \path[draw] (b2.south) -- (b5.north);
  \end{tikzpicture}
  \end{center}
  \end{minipage}
\end{frame}

\begin{frame}[fragile]
  \frametitle{``Lifting'' a function}
  \begin{code}
foo :: Int -> String
foo x
  | odd x = show $ (3 * x + 1) `div` 2
  | otherwise = show $ x `div` 2
  \end{code}
  \begin{minipage}[t][.4\textheight]{\linewidth}
  \begin{center}
  \tikzstyle{sqr} = [draw=black,
    minimum width=0.5cm,
    minimum height=0.5cm,
    node distance=1.3cm]
  \begin{tikzpicture}
    \node (f) {};
    \node[below of=f, node distance=1cm] (ff) {};
    \node[below of=ff, node distance=1cm] (fff) {};
    \node[left of=fff, node distance=1cm] (fffl) {};
    \node[right of=fff, node distance=1cm] (fffr) {};
    \path[draw, -latex] (fffl.west) -- node[above] {\icwtv{fmap foo}} (fffr.east);
    \node[sqr, left of=ff, node distance=3cm] (a1) {\icv{1}};
    \node[sqr, below left of=a1] (a2) {\icv{2}};
    \node[sqr, below right of=a1] (a3) {\icv{3}};
    \node[sqr, below left of=a2] (a4) {\icv{4}};
    \node[sqr, below right of=a2] (a5) {\icv{5}};
    \path[draw] (a1.south) -- (a2.north);
    \path[draw] (a1.south) -- (a3.north);
    \path[draw] (a2.south) -- (a4.north);
    \path[draw] (a2.south) -- (a5.north);
    \node[sqr, right of=ff, node distance=3cm] (b1) {\icv{"2"}};
    \node[sqr, below left of=b1] (b2) {\icv{"1"}};
    \node[sqr, below right of=b1] (b3) {\icv{"5"}};
    \node[sqr, below left of=b2] (b4) {\icv{"2"}};
    \node[sqr, below right of=b2] (b5) {\icv{"8"}};
    \path[draw] (b1.south) -- (b2.north);
    \path[draw] (b1.south) -- (b3.north);
    \path[draw] (b2.south) -- (b4.north);
    \path[draw] (b2.south) -- (b5.north);
  \end{tikzpicture}
  \end{center}
  \end{minipage}
\end{frame}

\begin{frame}[fragile]
  \frametitle{ap(ply)}
  \begin{itemize}
    \uncover<1->{\item what about functions with several arguments?}
    \uncover<5->{\item apply to the rescue!}
    \uncover<7->{\item also defined as an operator}
  \end{itemize}
  \begin{minipage}[t][.4\textheight]{\linewidth}
    \begin{onlyenv}<2>
      \begin{code}
(*\\*)
(*\\*)
fmap (+)        (Just 3) =
(*\\*)
      \end{code}
    \end{onlyenv}
    \begin{onlyenv}<3>
      \begin{code}
(*\\*)
(*\\*)
fmap (+)        (Just 3) = Just (3+)
(*\\*)
      \end{code}
    \end{onlyenv}
    \begin{onlyenv}<4>
      \begin{code}
(*\\*)
(*\\*)
fmap (+)        (Just 3) = Just (3+)
-- (*\cons{Just}*) (3+) :: (*\type{Maybe}*) ((*\type{Int}*) -> (*\type{Int}*))
      \end{code}
    \end{onlyenv}
    \begin{onlyenv}<5>
      \begin{code}
ap :: c (a -> b) -> c a -> c b
(*\\*)
fmap (+)        (Just 3) = Just (3+)
ap (Just (3+)) (Just 39) =
      \end{code}
    \end{onlyenv}
    \begin{onlyenv}<6>
      \begin{code}
ap :: c (a -> b) -> c a -> c b
(*\\*)
fmap (+)        (Just 3) = Just (3+)
ap (Just (3+)) (Just 39) = Just 42
      \end{code}
    \end{onlyenv}
    \begin{onlyenv}<7>
      \begin{code}
(<*>) :: c (a -> b) -> c a -> c b
(*\\*)
fmap (+)        (Just 3) = Just (3+)
Just (3+)  <*>   Just 39 = Just 42
      \end{code}
    \end{onlyenv}
    \begin{onlyenv}<8>
      \begin{code}
(<*>) :: c (a -> b) -> c a -> c b
(*\\*)
  (+)      <$>   Just 3  = Just (3+)
Just (3+)  <*>   Just 39 = Just 42
      \end{code}
    \end{onlyenv}
    \begin{onlyenv}<9>
      \begin{code}
     (+)       <$> Just 3  = Just (3+)
   Just (3+)   <*> Just 39 = Just 42
(*\\*)
(+) <$> Just 3 <*> Just 39 = Just 42
      \end{code}
    \end{onlyenv}
  \end{minipage}
\end{frame}

\begin{frame}[fragile]
  \frametitle{When to use ap}
  \begin{itemize}
    \item ``Wrapped'' values \textbf{and} functions
    \item Want to apply functions to values
  \end{itemize}
  \begin{minipage}[t][.3\textheight]{\linewidth}
  \begin{center}
  \begin{onlyenv}<1>
    \tikzstyle{sqr} = [draw=black,
      minimum width=0.5cm,
      minimum height=0.5cm,
      node distance=1.3cm]
    \begin{tikzpicture}
      \node (root) {};
      \node[sqr, below of=root, node distance=1cm] (a1) {\icv{a}};
      \node[sqr, below left of=a1] (a2) {\icv{a}};
      \node[sqr, below right of=a1] (a3) {\icv{a}};
      \path[draw] (a1.south) -- (a2.north);
      \path[draw] (a1.south) -- (a3.north);
      \node[sqr, right of=a1, node distance=3cm] (b1) {\icv{b}};
      \node[sqr, below left of=b1] (b2) {\icv{b}};
      \node[sqr, below right of=b1] (b3) {\icv{b}};
      \path[draw] (b1.south) -- (b2.north);
      \path[draw] (b1.south) -- (b3.north);
      \node[sqr, left of=a1, node distance=4cm] (f1) {\icwtv{a -> b}};
      \node[sqr, below left of=f1] (f2) {\icwtv{a -> b}};
      \node[sqr, below right of=f1] (f3) {\icwtv{a -> b}};
      \path[draw] (f1.south) -- (f2.north);
      \path[draw] (f1.south) -- (f3.north);
      \node[below of=a1, node distance=2cm] {\icwtv{Tree a}};
      \node[below of=b1, node distance=2cm] {\icwtv{Tree b}};
      \node[below of=f1, node distance=2cm] {\icwtv{Tree (a -> b)}};
      \node[right of=a1, node distance=1.5cm] (eq) {};
      \node[below of=eq, node distance=0.5cm] {\icwtv{=}};
      \node[left of=a1, node distance=1.8cm] (op) {};
      \node[below of=op, node distance=0.5cm] {\icwtv{<*>}};
    \end{tikzpicture}
  \end{onlyenv}
  \begin{onlyenv}<2>
    \tikzstyle{sqr} = [draw=black,
      minimum width=0.5cm,
      minimum height=0.5cm,
      node distance=1.3cm]
    \begin{tikzpicture}
      \node (root) {};
      \node[sqr, below of=root, node distance=1cm] (a1) {\icwtv{1}};
      \node[sqr, below left of=a1] (a2) {\icwtv{2}};
      \node[sqr, below right of=a1] (a3) {\icwtv{3}};
      \path[draw] (a1.south) -- (a2.north);
      \path[draw] (a1.south) -- (a3.north);
      \node[sqr, right of=a1, node distance=3cm] (b1) {\icwtv{4}};
      \node[sqr, below left of=b1] (b2) {\icwtv{4}};
      \node[sqr, below right of=b1] (b3) {\icwtv{9}};
      \path[draw] (b1.south) -- (b2.north);
      \path[draw] (b1.south) -- (b3.north);
      \node[sqr, left of=a1, node distance=4cm] (f1) {\icwtv{(*4)}};
      \node[sqr, below left of=f1] (f2) {\icwtv{(+2)}};
      \node[sqr, below right of=f1] (f3) {\icwtv{(^2)}};
      \path[draw] (f1.south) -- (f2.north);
      \path[draw] (f1.south) -- (f3.north);
      \node[below of=a1, node distance=2cm] {\icwtv{Tree Int}};
      \node[below of=b1, node distance=2cm] {\icwtv{Tree Int}};
      \node[below of=f1, node distance=2cm] {\icwtv{Tree (Int -> Int)}};
      \node[right of=a1, node distance=1.5cm] (eq) {};
      \node[below of=eq, node distance=0.5cm] {\icwtv{=}};
      \node[left of=a1, node distance=1.8cm] (op) {};
      \node[below of=op, node distance=0.5cm] {\icwtv{<*>}};
    \end{tikzpicture}
  \end{onlyenv}
  \end{center}
  \end{minipage}
\end{frame}

\begin{frame}[fragile]
  \frametitle{Using ap on lists}
  \begin{onlyenv}<1-2>
  \begin{code}
data ZipList a = ZipList [a]
funs = ZipList [(+1), (^2)]
vals = ZipList [2, 3, 4]
funs <*> vals = ZipList [3, 9]
  \end{code}
  \end{onlyenv}
  \begin{onlyenv}<2>
  \begin{code}
data AllList a = AllList [a]
funs = AllList [(+1), (^2)]
vals = AllList [2, 3, 4]
funs <*> vals = AllList [3, 4, 5, 4, 9, 16]
  \end{code}
  \end{onlyenv}
  \begin{onlyenv}<3>
  \begin{code}
newtype ZipList a = ZipList [a] -- newtype vs data
funs = ZipList [(+1), (^2)]
vals = ZipList [2, 3, 4]
funs <*> vals = ZipList [3, 9]
  \end{code}
  \begin{code}
-- [(*\tvar{a}*)] for cartesian product
funs = [(+1), (^2)]
vals = [2, 3, 4]
funs <*> vals = [3, 4, 5, 4, 9, 16]
  \end{code}
  \end{onlyenv}
\end{frame}

\begin{frame}[fragile]
  \frametitle{Applicative style}
  \begin{itemize}
    \item<1-> If we have a \icwtv{f :: a -> b -> c}, how do we get a \icwtv{c}?
    \item<2-> We need \icwtv{x :: a} and \icwtv{y :: b}
    \item<3-> What if we have \icwtv{wx :: W a} and \icwtv{wy :: W b}?\\(for some ``wrapper'' \icwtv{W})
    \item<4-> \icwtv{fmap} and \icwtv{ap} give us \icwtv{W c}!
  \end{itemize}
  \begin{minipage}[t][.3\textheight]{\linewidth}
  \begin{onlyenv}<2-3>
  \begin{code}
f      x      y -- result :: (*\tvar{c}*)
(*\\*)
  \end{code}
  \end{onlyenv}
  \begin{onlyenv}<4>
  \begin{code}
f      x      y -- result ::   (*\tvar{c}*)
f <$> wx <*> wy -- result :: W (*\tvar{c}*)
  \end{code}
  \end{onlyenv}
  \end{minipage}
\end{frame}

\begin{frame}[fragile]
  \frametitle{bind}
  \begin{itemize}
    \uncover<1->{\item what about chaining functions?}
    \uncover<8->{\item bind to the rescue!}
    \uncover<11->{\item of course, also has an operator}
  \end{itemize}
  \begin{onlyenv}<2>
    \begin{code}
div2 :: (*\type{Int}*) -> (*\type{Maybe}*) (*\type{Int}*)
(*\\*)
(*\\*)
    \end{code}
  \end{onlyenv}
  \begin{onlyenv}<3>
    \begin{code}
div2 :: (*\type{Int}*) -> (*\type{Maybe}*) (*\type{Int}*)
div2 42 = (*\cons{Just}*) 21
div2 21 = (*\cons{Nothing}*)
    \end{code}
  \end{onlyenv}
  \begin{onlyenv}<4-7>
    \begin{code}[given]
div2 :: (*\type{Int}*) -> (*\type{Maybe}*) (*\type{Int}*)
    \end{code}
  \end{onlyenv}
  \begin{onlyenv}<4>
    \begin{code}
div4 :: (*\type{Int}*) -> (*\type{Maybe}*) (*\type{Int}*)
div4 x =
(*\\*)
(*\\*)
    \end{code}
  \end{onlyenv}
  \begin{onlyenv}<5>
    \begin{code}
div4 :: (*\type{Int}*) -> (*\type{Maybe}*) (*\type{Int}*)
div4 x = let y = div2 x
         in
(*\\*)
    \end{code}
  \end{onlyenv}
  \begin{onlyenv}<6>
    \begin{code}
div4 :: (*\type{Int}*) -> (*\type{Maybe}*) (*\type{Int}*)
div4 x = let y = div2 x
         in fmap div2 y
(*\\*)
    \end{code}
  \end{onlyenv}
  \begin{onlyenv}<7>
    \begin{code}
div4 :: (*\type{Int}*) -> (*\type{Maybe}*) (*\type{Int}*)
div4 x = let y = div2 x -- (*\type{Maybe}*) (*\type{Int}*)
         in fmap div2 y -- (*\type{Maybe}*) ((*\type{Maybe}*) (*\type{Int}*))
(*\\*)
    \end{code}
  \end{onlyenv}
  \begin{onlyenv}<8-10>
    \begin{code}[bind]
bind :: (a -> c b) -> c a -> c b
    \end{code}
  \end{onlyenv}
  \begin{onlyenv}<9>
    \begin{code}
div2, div4 :: (*\type{Int}*) -> (*\type{Maybe}*) (*\type{Int}*)
div4 x = let y = div2 x -- (*\type{Maybe}*) (*\type{Int}*)
         in bind div2 y -- (*\type{Maybe}*) (*\type{Int}*)
// a = b = Int, c = Maybe
    \end{code}
  \end{onlyenv}
  \begin{onlyenv}<10>
    \begin{code}
div2, div4 :: (*\type{Int}*) -> (*\type{Maybe}*) (*\type{Int}*)
div4 x = bind div2 $ div2 x
(*\\*)
(*\\*)
    \end{code}
  \end{onlyenv}
  \begin{onlyenv}<11->
    \begin{code}[bind operator]
(>>=) :: c a -> (a -> c b) -> c b
    \end{code}
  \end{onlyenv}
  \begin{onlyenv}<11>
    \begin{code}
div2, div4 :: (*\type{Int}*) -> (*\type{Maybe}*) (*\type{Int}*)
div4 x = div2 x >>= div2
(*\\*)
(*\\*)
    \end{code}
  \end{onlyenv}
  \begin{onlyenv}<12>
    \begin{code}
div2, div4, div8 :: (*\type{Int}*) -> (*\type{Maybe}*) (*\type{Int}*)
div4 x = div2 x >>= div2
div8 x = div2 x >>= div2 >>= div2
(*\\*)
    \end{code}
  \end{onlyenv}
  \begin{onlyenv}<13>
    \begin{code}
div2, div4, div8, div16 :: (*\type{Int}*) -> (*\type{Maybe}*) (*\type{Int}*)
div4  x = div2 x >>= div2
div8  x = div2 x >>= div2 >>= div2
div16 x = div2 x >>= div2 >>= div2 >>= div2
    \end{code}
  \end{onlyenv}
\end{frame}

\begin{frame}[fragile]
  \frametitle{bind - continued}
  \begin{code}
getUser        :: (*\type{Id}*)   -> (*\type{Maybe}*) (*\type{User}*)
getNextOfKin   :: (*\type{User}*) -> (*\type{Maybe}*) (*\type{Id}*)
getPhoneNumber :: (*\type{User}*) -> (*\type{Maybe}*) (*\type{PhoneNumber}*)
(>>=) :: c a -> (a -> c b) -> c b -- c == Maybe
  \end{code}
  \begin{onlyenv}<1>
    \begin{code}
getNextOfKinPhoneNumber :: (*\type{Id}*) -> (*\type{Maybe}*) (*\type{PhoneNumber}*)
getNextOfKinPhoneNumber uid =
(*\\*)
(*\\*)
(*\\*)
(*\\*)
    \end{code}
  \end{onlyenv}
  \begin{onlyenv}<2>
    \begin{code}
getNextOfKinPhoneNumber :: (*\type{Id}*) -> (*\type{Maybe}*) (*\type{PhoneNumber}*)
getNextOfKinPhoneNumber uid =
      getUser uid     -- (*\type{Maybe}*) (*\type{User}*)
(*\\*)
(*\\*)
(*\\*)
    \end{code}
  \end{onlyenv}
  \begin{onlyenv}<3>
    \begin{code}
getNextOfKinPhoneNumber :: (*\type{Id}*) -> (*\type{Maybe}*) (*\type{PhoneNumber}*)
getNextOfKinPhoneNumber uid =
      getUser uid     -- (*\type{Maybe}*) (*\type{User}*)
  >>= getNextOfKin    -- (*\type{Maybe}*) (*\type{Id}*)
(*\\*)
(*\\*)
    \end{code}
  \end{onlyenv}
  \begin{onlyenv}<4>
    \begin{code}
getNextOfKinPhoneNumber :: (*\type{Id}*) -> (*\type{Maybe}*) (*\type{PhoneNumber}*)
getNextOfKinPhoneNumber uid =
      getUser uid     -- (*\type{Maybe}*) (*\type{User}*)
  >>= getNextOfKin    -- (*\type{Maybe}*) (*\type{Id}*)
  >>= getUser         -- (*\type{Maybe}*) (*\type{User}*)
(*\\*)
    \end{code}
  \end{onlyenv}
  \begin{onlyenv}<5>
    \begin{code}
getNextOfKinPhoneNumber :: (*\type{Id}*) -> (*\type{Maybe}*) (*\type{PhoneNumber}*)
getNextOfKinPhoneNumber uid =
      getUser uid     -- (*\type{Maybe}*) (*\type{User}*)
  >>= getNextOfKin    -- (*\type{Maybe}*) (*\type{Id}*)
  >>= getUser         -- (*\type{Maybe}*) (*\type{User}*)
  >>= getPhoneNumber  -- (*\type{Maybe}*) (*\type{User}*)
    \end{code}
  \end{onlyenv}
\end{frame}

\begin{frame}
  \frametitle{When to use bind}
  \begin{itemize}
    \item<1-> ``Collapse'' the ``wrapping''
    \item<2-> ``Decide'' on ``wrapping''
    \item<3-> Parsers for context free languages vs context sensitive ones
  \end{itemize}
\end{frame}

\def\Sep{~~~~~&~~~~~}
\begin{frame}
  \frametitle{The typeclasses}
  \begin{onlyenv}<1-2>
    \begin{center}
      \pc{fmap :: ~~\icwtv{(a -> b) -> c a -> c b}\\}
      \pc{ap~~ :: \icwtv{c (a -> b) -> c a -> c b}\\}
      \pc{bind :: \icwtv{(a -> c b) -> c a -> c b}\\}
      \uncover<2>{~\\But what about the typeclasses?}
    \end{center}
  \end{onlyenv}
  \begin{onlyenv}<3-7,9->
    \begin{center}
      ~\\
      \begin{tabular}{ c l c l c }
        \uncover<3->{                  \Sep\pc{fmap}\Sep$\rightarrow$\Sep\upc<4->{Functor    }\Sep                  \\}
        \uncover<5->{$^\downrcurvearrow$\Sep\pc{ap}  \Sep$\rightarrow$\Sep\upc<6->{Applicative}\Sep$^\downlcurvearrow$\\}
        \uncover<7->{$^\downrcurvearrow$\Sep\pc{bind}\Sep$\rightarrow$\Sep\upc<8->{Monad}      \Sep$^\downlcurvearrow$\\}
      \end{tabular}
    \end{center}
  \end{onlyenv}
  \begin{onlyenv}<8>
    \begin{minipage}[c][.8\textheight]{\linewidth}
      \begin{center}
        {\fontsize{150}{60}\selectfont MONAD}\\
      \end{center}
    \end{minipage}
  \end{onlyenv}
\end{frame}

\begin{frame}[fragile]
  \frametitle{Monad}
  \begin{minipage}[t][.5\textheight]{\linewidth}
    \begin{center}
      \begin{code}
class (*\type{Functor}*) f where
  fmap   ::   (b -> a) -> f b -> f a
class (*\type{Functor}*) f => (*\type{Applicative}*) f where
  pure   ::                 a -> f a
  (<*>)  :: f (b -> a) -> f b -> f a
class (*\type{Applicative}*) m => (*\type{Monad}*) m where
  return ::                 a -> m a
  (>>=)  :: m b -> (b -> m a) -> m a
      \end{code}
    \end{center}
  \end{minipage}
  \begin{minipage}[t][.3\textheight]{\linewidth}
    \begin{center}
      The basic building blocks
    \end{center}
  \end{minipage}
\end{frame}


\section{Codelab - part 2}

\begin{frame}
  \frametitle{Codelab :: Section 3}
  \begin{description}
    \item[directory]: \ic{01_mastermind/}
    \item[change]: \ic{src/ErrorOr.hs} \\
      (replace \ic{codelab} with implementation)
    \item[check with]: \ic{make ARGS="check 3"}
  \end{description}
\end{frame}


\section{IO}

\begin{frame}
  \frametitle{We still don't know}
  \begin{center}
    \begin{minipage}[c]{.8\linewidth}
      \begin{itemize}
        \item \st{how to extend our data types}
        \item \st{how to express type constraints}
        \item \st{how to chain contextual functions}
        \item how to use IO
      \end{itemize}
    \end{minipage}
  \end{center}
\end{frame}

\begin{frame}[fragile]
  \frametitle{Using IO}
  \begin{onlyenv}<1-4>
    \begin{code}
getFirstName :: IO String
getLastName  :: IO String
greetUser    :: String -> String -> IO ()
(*\\*)
    \end{code}
  \end{onlyenv}
  \begin{onlyenv}<5->
    \begin{code}
getFirstName :: IO String
getLastName  :: IO String
greetUser    :: String -> String -> IO ()
(>>=)        :: (*\tvar{c}*) (*\tvar{a}*) -> ((*\tvar{a}*) -> (*\tvar{c}*) (*\tvar{b}*)) -> (*\tvar{c}*) (*\tvar{b}*)
    \end{code}
  \end{onlyenv}
  \begin{onlyenv}<2>
    \begin{code}
main         :: IO ()
(*\\*)
(*\\*)
(*\\*)
(*\\*)
    \end{code}
  \end{onlyenv}
  \begin{onlyenv}<3>
    \begin{code}
main         :: IO ()
main =
(*\\*)
(*\\*)
(*\\*)
    \end{code}
  \end{onlyenv}
  \begin{onlyenv}<4>
    \begin{code}
main         :: IO ()
main =
  getFirstName
(*\\*)
(*\\*)
    \end{code}
  \end{onlyenv}
  \begin{onlyenv}<5>
    \begin{code}
main         :: IO ()
main =
  getFirstName >>=
(*\\*)
(*\\*)
    \end{code}
  \end{onlyenv}
  \begin{onlyenv}<6>
    \begin{code}
main         :: IO ()
main =
  getFirstName >>= (\firstname -> ...)
(*\\*)
(*\\*)
    \end{code}
  \end{onlyenv}
  \begin{onlyenv}<7>
    \begin{code}
main         :: IO ()
main =
  getFirstName >>= (\firstname ->
    ...)
(*\\*)
    \end{code}
  \end{onlyenv}
  \begin{onlyenv}<8>
    \begin{code}
main         :: IO ()
main =
  getFirstName >>= (\firstname ->
    getLastName ...)
(*\\*)
    \end{code}
  \end{onlyenv}
  \begin{onlyenv}<9>
    \begin{code}
main         :: IO ()
main =
  getFirstName >>= (\firstname ->
    getLastName >>= (\lastname ->
      ...))
    \end{code}
  \end{onlyenv}
  \begin{onlyenv}<10>
    \begin{code}
main         :: IO ()
main =
  getFirstName >>= (\firstname ->
    getLastName >>= (\lastname ->
      greetUser firstname lastname))
    \end{code}
  \end{onlyenv}
  \begin{onlyenv}<11>
    \begin{code}
main         :: IO ()
main =
  getFirstName >>= \firstname ->
    getLastName >>= \lastname ->
      greetUser firstname lastname
    \end{code}
  \end{onlyenv}
\end{frame}


\section{Monadic syntax}

\begin{frame}[fragile]
\frametitle{Do notation}
  \begin{onlyenv}<1>
    \begin{code}
main :: IO ()
main =
  getFirstName >>= \firstname ->
    getLastName >>= \lastname ->
      greetUser firstname lastname
    \end{code}
  \end{onlyenv}
  \begin{onlyenv}<2>
    \begin{code}[important parts]
main :: IO ()
main =
  getFirstName      firstname
    getLastName      lastname
      greetUser firstname lastname
    \end{code}
  \end{onlyenv}
  \begin{onlyenv}<3>
    \begin{code}[what we want]
main :: IO ()
main =
  getFirstName -> firstname
    getLastName -> lastname
      greetUser firstname lastname
    \end{code}
  \end{onlyenv}
  \begin{onlyenv}<4>
    \begin{code}[align spaces]
main :: IO ()
main =
  getFirstName -> firstname
  getLastName -> lastname
  greetUser firstname lastname
    \end{code}
  \end{onlyenv}
  \begin{onlyenv}<5->
    \begin{code}
main :: IO ()
main = do
  firstname <- getFirstName
  lastname <- getLastName
  greetUser firstname lastname
    \end{code}
  \end{onlyenv}
  \uncover<6->{\begin{center}``Haskell is the best imperative language''\end{center}}
\end{frame}

\begin{frame}[fragile]
\frametitle{Do notation}
  \begin{onlyenv}<1>
    \begin{code}
getNextOfKinPhoneNumber :: (*\type{Id}*) -> (*\type{Maybe}*) (*\type{PhoneNumber}*)
getNextOfKinPhoneNumber uid =
      getUser uid
  >>= getNextOfKin
  >>= getUser
  >>= getPhoneNumber
    \end{code}
  \end{onlyenv}
  \begin{onlyenv}<2>
    \begin{code}
getNextOfKinPhoneNumber :: (*\type{Id}*) -> (*\type{Maybe}*) (*\type{PhoneNumber}*)
getNextOfKinPhoneNumber uid =
  getUser uid >>= \user ->
    getNextOfKin user >>= \nextOfKinId ->
      getUser nextOfKinId >>= \nextOfKinUser ->
        getPhoneNumber nextOfKinUser
    \end{code}
  \end{onlyenv}
  \begin{onlyenv}<3>
    \begin{code}[important parts]
getNextOfKinPhoneNumber :: (*\type{Id}*) -> (*\type{Maybe}*) (*\type{PhoneNumber}*)
getNextOfKinPhoneNumber uid =
  getUser uid      user
    getNextOfKin user      nextOfKinId
      getUser nextOfKinId      nextOfKinUser
        getPhoneNumber nextOfKinUser
    \end{code}
  \end{onlyenv}
  \begin{onlyenv}<4>
    \begin{code}
getNextOfKinPhoneNumber :: (*\type{Id}*) -> (*\type{Maybe}*) (*\type{PhoneNumber}*)
getNextOfKinPhoneNumber uid = do
  user <- getUser uid
  nextOfKinId <- getNextOfKin user
  nextOfKinUser <- getUser nextOfKinId
  getPhoneNumber nextOfKinUser
    \end{code}
  \end{onlyenv}
\end{frame}

\begin{frame}[fragile]
\frametitle{Do notation}
  \begin{code}
getNextOfKinPhoneNumber :: (*\type{Id}*) -> (*\type{Maybe}*) (*\type{PhoneNumber}*)
getNextOfKinPhoneNumber uid = do
  user <- getUser uid
  nextOfKinId <- getNextOfKin user
  nextOfKinUser <- getUser nextOfKinId
  getPhoneNumber nextOfKinUser
  \end{code}
  \begin{code}
main :: IO ()
main = do
  firstname <- getFirstName
  lastname <- getLastName
  greetUser firstname lastname
  \end{code}
\end{frame}

\begin{frame}
  \frametitle{The power of monads}
  \begin{itemize}[<+->]
    \item Programmable semicolon
    \item Welcome to Haskell. We have:
      \begin{itemize}[<+->]
        \item optional chaining     (using \ic{>>=}/monads)
        \item input/output          (using \ic{>>=}/monads)
        \item multithreading        (using \ic{>>=}/monads)
        \item error handling        (using \ic{>>=}/monads)
        \item mutable state         (using \ic{>>=}/monads)
        \item list comprehension    (using \ic{>>=}/monads)
        \item randomness            (using \ic{>>=}/monads)
        \item async/await           (using \ic{>>=}/monads)
      \end{itemize}
  \end{itemize}
\end{frame}

\begin{frame}[fragile]
  \frametitle{Bonus: the power of monads}
  \begin{onlyenv}<1-9>
    \begin{center}
      \pc{mapM :: \icwtv{(Monad m) => (a -> m b) -> [a] -> m [b]}\\}
    \end{center}
  \end{onlyenv}
  \begin{onlyenv}<2-3>
    \begin{code}
div2 :: Int -> Maybe Int
div2 x
  | even x    = Just (x `div` 2)
  | otherwise = Nothing
    \end{code}
  \end{onlyenv}
  \begin{onlyenv}<3>
    \begin{code}[eval]
Prelude> mapM div2 [2,4,6,8]
Just [1,2,3,4]
Prelude> mapM div2 [2,3,4,5]
Nothing
    \end{code}
  \end{onlyenv}
  \begin{onlyenv}<4-5>
    \begin{code}
div2 :: Int -> Either String Int
div2 x
  | even x    = Right (x `div` 2)
  | otherwise = Left ("Odd: " ++ show x)
    \end{code}
  \end{onlyenv}
  \begin{onlyenv}<5>
    \begin{code}[eval]
Prelude> mapM div2 [2,4,6,8]
Right [1,2,3,4]
Prelude> mapM div2 [2,3,4,5]
Left "Odd: 3"
    \end{code}
  \end{onlyenv}
  \begin{onlyenv}<6-7>
    \begin{code}
div2 :: Int -> IO Int
div2 x
  | even x    = return (x `div` 2)
  | otherwise = putStrLn ("Odd: " ++ show x) >>
                return 0
    \end{code}
  \end{onlyenv}
  \begin{onlyenv}<7>
    \begin{code}[eval]
Prelude> mapM div2 [2,4,6,8]
[1,2,3,4]
Prelude> mapM div2 [2,3,4,5]
Odd: 3
Odd: 5
[1,0,2,0]
    \end{code}
  \end{onlyenv}
  \begin{onlyenv}<8-9>
    \begin{code}
div2 :: Int -> Writer [String] Int
div2 x
  | even x    = return (x `div` 2)
  | otherwise = tell ["Odd: " ++ show x] >>
                return 0
    \end{code}
  \end{onlyenv}
  \begin{onlyenv}<9>
    \begin{code}[eval]
Prelude> mapM div2 [2,4,6,8]
WriterT (Identity ([1,2,3,4],[]))
Prelude> mapM div2 [2,3,4,5]
WriterT (Identity ([1,0,2,0],["Odd: 3","Odd: 5"]))
    \end{code}
  \end{onlyenv}
\end{frame}

\begin{frame}
  \frametitle{The end of the theoretical part!}
  \begin{center}
    Questions?
  \end{center}
\end{frame}


% Links

\begin{frame}
  \frametitle{Links}
  \begin{itemize}
    \item \href{http://adit.io/posts/2013-04-17-functors,_applicatives,_and_monads_in_pictures.html}{adit.io} (functors, applicatives, and monads in pictures)
    \item \href{http://dev.stephendiehl.com/hask/}{dev.stephendiehl.com/hask/} (what I wish I knew)
    \item \href{https://www.forbes.com/sites/quora/2018/07/27/why-purely-functional-programming-is-a-great-idea-with-a-misleading-name/}{Forbes - Why Purely Functional Programming Is A Great Idea With A Misleading Name}
  \end{itemize}
\end{frame}


\section{Codelab - part 3}

\begin{frame}
  \frametitle{Codelab :: Section 4}
  \begin{description}
    \item[directory]: \ic{01_mastermind/}
    \item[change]: \ic{src/Code.hs} \\
      (replace \ic{codelab} with implementation)
    \item[check with]: \ic{make ARGS="check 4"}
  \end{description}
\end{frame}

\begin{frame}
  \frametitle{Codelab :: Section 5}
  \begin{description}
    \item[directory]: \ic{01_mastermind/}
    \item[change]: \ic{src/Do.hs} \\
      (replace \ic{codelab} with implementation)
    \item[check with]: \ic{make ARGS="check 5"}
  \end{description}
\end{frame}

\begin{frame}
  \frametitle{Codelab :: The game}
  \begin{description}
    \item[directory]: \ic{01_mastermind/}
    \item[play]: \ic{make ARGS=play}
    \item[solve (AI)]: \ic{make ARGS=solve}
  \end{description}
\end{frame}


\section{The end}

\begin{frame}
  \frametitle{The end!}
  \begin{center}
    Questions?
  \end{center}
\end{frame}

%%%%%%%%%%%%%%%%%%%%%%%%%%%%%%%%%%%%%%%%%%%%%%%%%%%%%%%%%%%%%%%%%%%%%%%%%%%%
%% End

\end{document}
